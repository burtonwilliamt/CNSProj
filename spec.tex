\documentclass{article}
\usepackage{amsmath}
\usepackage{amssymb}
\usepackage{mathtools}
\usepackage{pgfplots}
\usepackage[letterpaper, portrait, margin=0.7in]{geometry}
\pgfplotsset{compat=1.13}
\usepackage{graphicx}
\graphicspath{ {./} }
\DeclarePairedDelimiter\abs{\lvert}{\rvert}
\title{Crypto 1 Project}
\author{
  William Burton\\
  \texttt{burtow@rpi.edu}
  \and
  Greg Cowan\\
  \texttt{cowang@rpi.edu}
}
\date{December 7th, 2017}
\begin{document}
\maketitle
\begin{flushleft}


\section{Requirements}
Alice and Bob each have m files, each file is $>10$MB. They want to find out
how many files they have in common and which ones are shared. \\
They start without a shared key, and they communicate over an insecure channel
(e.g. LAN, WAN, or Internet). Alice and Bob are semi-honest, so our
protocol should handle that. \\
An eavesdropper should not be able to alter their communication or learn
something about the transmitted messages. Alice and Bob should not learn
anything else other than the number and identities of the common files. \\
We can use: collision resistant hash functions, existentially unforgeable MACs
and digital signatures, and CPA-secure public-key and symmetric-key encryption.


\section{Protocol}
\begin{enumerate}
  \item Ahead of time, a private-key, public-key pair is generated for both Alice and Bob.
  \item Alice is initialized with Bob's public key, and her private key. Bob is initialized with Alice's public key and his private key.
  In this case, these are shared by a trusted-courier, the user of the simulation. But they could be shared another secure way in a real application.
  \item Alice and Bob use a Hash-and-sign signature sheme, like RSA-FDH, to transform the insecure channel to an authenticated one.
  \item Alice and Bob transform the authenticated channel into an encrypted one using the El Gamal scheme.
  \item Now, they can start the application-specific protocol.
    \begin{enumerate}
      \item Alice and Bob hash one copy of all their files.
      \item Alice and Bob encrypt their hashes with \textit{their public} keys.
      \item Alice and Bob send those encrypted hashes to the other one.
      \item Bob and Alice encrypt their file's hashes with the other's public-key.
      \item If an encryption matches, then they share that file.
    \end{enumerate}
\end{enumerate}


\end{flushleft}
\end{document}
